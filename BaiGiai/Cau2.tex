\section{Câu 2}
Tìm công thức liên hệ giữa $I_L$ và $V_i$ (hình a) và giữa $I_L$ và $V_1 - V_2$ (hình b).

a. Giả sử OPAMP đang dùng là lí tưởng.

b. Giả sử cả 2 OPAMP có $V_{io}=100\,\mathrm{\mu V}$, $I_{io}=4\,\mathrm{nA}$, $I_{ib}=18\,\mathrm{nA}$.

Tìm ảnh hưởng của điều kiện không lý tưởng của OPAMP lên ngõ ra $I_L$.

\begin{figure}[H]
    \centering
    \includegraphics[scale=0.6]{image/De_C2.png} 
\end{figure}

\begin{center}
\textbf{Bài giải}
\end{center}


Đối với hình a:\\

\begin{figure}[H]
    \centering
    \includegraphics[scale=0.2]{image/C2_a1.png} 
\end{figure}

Xét nút $V^{-}$:
\begin{flalign*}
& \frac{V_i - V^-}{R_A} = -\frac{V_{o1} - V^-}{R_1} \quad(\text{theo chiều dòng}) && \\
& \Rightarrow V^- = \frac{V_i + V_{o1}}{2} \quad(\text{nếu }R_A=R_1) && 
\end{flalign*}

Tại $V^{+}$ (phân chia điện áp bởi hai trở bằng nhau):
\[
V^{+}=V_{o2}\cdot\frac{R_2}{R_2+R_2}=\frac{V_{o2}}{2}.
\]

Vì op-amp ở vùng phản hồi âm nên $V^{+}=V^{-}$. Do đó:
\[
\frac{V_{o2}}{2}=\frac{V_i+V_{o1}}{2}\quad\Rightarrow\quad V_i=V_{o2}-V_{o1}.
\]

Dòng tải theo $R_L$ (ghi theo ghi chú):
\[
I_L=\frac{V_i}{R_L}=\frac{V_{o1}}{R_L}=\frac{V_{o2}-V_i}{R_L}.
\]
\fbox{$\Rightarrow I_L=\frac{V_{o2}-V_i}{R_L}$}
\\
Đối với hình b:\\
\begin{figure}[H]
    \centering
    \includegraphics[scale=0.2]{image/C2_a2.jpg} 
\end{figure}

Xét OPAMP thứ nhất:
\begin{align*}
    \frac{V_1 - V_i}{R_1} &= \frac{V_{o1} - V_1}{R_2} \\
    V_1 \left(\frac{1}{R_1} + \frac{1}{R_2}\right) &= \frac{V_i}{R_1} + \frac{V_{o1}}{R_2}
\end{align*}

Xét OPAMP thứ hai:
\begin{align*}
    \frac{V_2 - V_i}{R_1} &= \frac{V_{o2} - V_2}{R_2} \\
    V_2 \left(\frac{1}{R_1} + \frac{1}{R_2}\right) &= \frac{V_i}{R_1} + \frac{V_{o2}}{R_2}
\end{align*}

Do đó:
\begin{align*}
    \frac{V_1}{R_1} + \frac{V_{o1}}{R_2} &= \frac{V_2}{R_1} + \frac{V_{o2}}{R_2} \\
    \frac{V_1 - V_2}{R_1} &= \frac{V_{o2} - V_{o1}}{R_2}
\end{align*}

Dòng điện qua tải:
\fbox{$\displaystyle
I_L = \frac{V_{o1} - V_{o2}}{R_x} = -\frac{(V_1 - V_2)R_2}{R_x \cdot R_1}
$}
\\[1em]
Câu b)\\
Đối với hình a:\\
\begin{figure}[H]
    \centering
    \includegraphics[scale=0.3]{image/C2_b1.png} 
\end{figure}

\begin{flalign*}
& V^- = V_{o1}\cdot\frac{R_1}{R_A + R_1} = \frac{V_{o1}}{2} && \\[1ex]
& V^+ = V_{o2} && \\[1ex]
& V^- = V^+ \Rightarrow \frac{V_{o1}}{2} = V_{o2} \Rightarrow V_{o1} = 2V_{o2} &&
\end{flalign*}

Xét tầng 2:\\
\begin{flalign*}
& V^+ = V_{o1} + V_{o2} = 3V_{o2} && \\[1ex]
& I_L = \frac{V^-}{R_L} = \frac{V^+}{R_L} = \frac{3V_{o2}}{R_L} &&
\end{flalign*}

\begin{flalign*}
\fbox{$\displaystyle I_L = \frac{3V_{o2}}{R_L}$}
\end{flalign*}
\\[1em]
Xét ảnh hưởng của dòng:\\
\begin{figure}[H]
    \centering
    \includegraphics[scale=0.4]{image/C2_b3.png} 
\end{figure}

Tại dòng của $I_b$ và $I_{os}$:
\begin{flalign*}
& \begin{cases}
I_b = \frac{I_1 + I_2}{2} \\[1ex]
I_{os} = |I_1 - I_2|
\end{cases} \Rightarrow
\begin{cases}
I^+ = 20\ \text{nA} \\
I^- = 16\ \text{nA} \\
I^+ = 16\ \text{nA} \\
I^- = 20\ \text{nA}
\end{cases} &&
\end{flalign*}

Ta có:
\begin{flalign*}
& V_A = -I^+ \cdot R_2//R_2 = -I^+ \cdot \frac{R_2}{2} && \\[2ex]
& I_1 = I^- + I_2 && \\[1ex]
& \Rightarrow I_1 = I^- + \frac{V_A}{R_1} && \\[1ex]
& = I^- + \frac{-I^+ \cdot R_2}{2R_1} && \\[2ex]
& \Delta V_1 - V_A = I_1 \cdot R_2 && \\[1ex]
& \Rightarrow \Delta V_1 = \left(I^- - \frac{I^+ \cdot R_2}{2R_1}\right)\cdot R_2 + V_A && \\[2ex]
& = \left(I^- - \frac{I^+ \cdot R_2}{2R_1}\right)\cdot R_2 - I^+ \cdot \frac{R_2}{2} && \\[2ex]
& \Rightarrow \Delta I_{L1} = \frac{\Delta V_1}{R_L} = \frac{\left(I^- - \frac{I^+\cdot R_2}{2R_1}\right)\cdot R_2 - I^+\cdot\frac{R_2}{2}}{R_L} &&
\end{flalign*}

\textbf{Xét tầng 2:}
\begin{flalign*}
& \Delta V_2 = (R_L\parallel R_3)\cdot (-I^-) && \\[2ex]
& \Delta I_{L2} = \frac{R_L\cdot R_3}{(R_L + R_3)\cdot R_L}\cdot (-I^-) && \\[2ex]
& \Delta I_{L2} = \frac{R_3}{R_L + R_3}\cdot (-I^-) && \\[3ex]
& \Rightarrow \Delta I = \Delta I_{L1} + \Delta I_{L2} = \frac{\left(I^- - \frac{I^+\cdot R_2}{2R_1}\right)\cdot R_2 - I^+\cdot\frac{R_2}{2}}{R_L} - \frac{R_3}{R_L + R_3}\cdot I^- && \\[2ex]
& \fbox{$\displaystyle \Delta I = \frac{\left(I^- - \frac{I^+\cdot R_2}{2R_1}\right)\cdot R_2 - I^+\cdot\frac{R_2}{2}}{R_L} - \frac{R_3}{R_L + R_3}\cdot I^-$}
\end{flalign*}
\\[1em]

Đối với hình b:\\
Xét ảnh hưởng do Vos:\\
\begin{figure}[H]
    \centering
    \includegraphics[scale=0.4]{image/C2_b4.png} 
\end{figure}

Xét $0 = V_{os} - \frac{V_{o1} - V_{o2}}{R_2}$:
\begin{flalign*}
& -V_{os}\cdot R_2 = R_1V_{os} - R_1V_{o1} && \\[1ex]
& \Rightarrow V_{o1} = V_{os}\cdot(R_1 + R_2) && \\[2ex]
& \Rightarrow \text{Ảnh hưởng của }V_{os}\text{ gây ra lỗi }\Delta V_{o1} = V_{os}(R_1 + R_2) && \\[2ex]
& \boxed{\Rightarrow \Delta I_L = \frac{\Delta V_{o1}}{R_x + Z_L} = \frac{V_{os}(R_1 + R_2)}{R_x + Z_L}} &&
\end{flalign*}
\\[1em]
\textbf{Khi ảnh hưởng của $I_b$ và $I_{os}$}\\
\textbf{Xét tầng 1:}
\begin{flalign*}
& V^+ = -I^+\cdot R_1\parallel R_2 && \\[2ex]
& I_1 = \frac{V^-}{R_1} = \frac{V^+}{R_1} = -\frac{I^+\cdot R_1\parallel R_2}{R_1} && \\[2ex]
& I_2 = I_1 + I^- && \\[2ex]
& \Rightarrow I_2 = -\frac{I^+\cdot R_1\parallel R_2}{R_1} + I^- && \\[2ex]
& \Delta V_{o1} - V^- = I_2\cdot R_2 && \\[2ex]
& \Rightarrow \Delta V_{o1} = \left(-\frac{I^+\cdot R_1\parallel R_2}{R_1} + I^-\right)\cdot R_2 + (-I^+\cdot R_1\parallel R_2) && \\[2ex]
& I_L\text{ do }\Delta V_{o1} = \frac{|\Delta V_{o1}|}{Z_L + R_x} && \\[2ex]
& \Rightarrow \Delta I_{L1} = \frac{\Delta V_{o1}}{Z_L + R_x} = \frac{\left(-\frac{I^+\cdot R_1\parallel R_2}{R_1} + I^-\right)\cdot R_2 + (-I^+\cdot R_1\parallel R_2)}{Z_L + R_x} &&
\end{flalign*}

\textbf{Xét tầng 2:}
\begin{flalign*}
& \Delta I_{L2} = \frac{R_L}{Z_L + R_L}\cdot (-I^-) && \\[3ex]
& \Rightarrow \Delta I_L = \Delta I_{L1} + \Delta I_{L2} = \frac{\left(-\frac{I^+\cdot R_1\parallel R_2}{R_1} + I^-\right)\cdot R_2 + (-I^+\cdot R_1\parallel R_2)}{Z_L + R_x} + \frac{R_L}{Z_L + R_L}\cdot (-I^-) &&
\end{flalign*}
\begin{center}
\fbox{$\displaystyle \Delta I_L = \frac{\left(-\frac{I^+\cdot R_1\parallel R_2}{R_1} + I^-\right)\cdot R_2 + (-I^+\cdot R_1\parallel R_2)}{Z_L + R_x} + \frac{R_L}{Z_L + R_L}\cdot (-I^-)$}
\end{center}
