\section{Câu 1}
Thiết kế mạch thực hiện hàm: $v_o = 2V_1 + 3V_2 -6V_3$ với yêu cầu chỉ được sử dụng 2 OPAMP.\\
a. Giả sử các OPAMP sử dụng là lí tưởng.\\
b. Giả sử cả 2 OPAMP có $V_{io}=100\mu V, I_{io}=4nA, I_{ib}=18nA$.\\
Tìm ảnh hưởng của điều kiện không lý tưởng của OPAMP lên ngõ ra $V_o$.\\
\begin{center}
\textbf{Bài giải}
\end{center}
\begin{figure}[H]
	\centering
	\includegraphics[scale=0.9]{image/C1_a_BT.png}
\end{figure}
a. Giả sử các OPAMP sử dụng là lí tưởng.\\
Ta có:
\[
\left\{
\begin{aligned}
v_{o1} &= -\dfrac{R_{f1}}{R_1}V_1 - \dfrac{R_{f1}}{R_2}V_2 &= -2V_1-3V_2\\
v_o &= -\dfrac{R_{f2}}{R_3}V_3 - \dfrac{R_{f2}}{R4}v_{o1} &= -6V_3-1v_{o1}
\end{aligned}
\right.
\]
Chọn $\boxed{R_{f1}=10K,\; R_{f2}=18K \;\rightarrow\; R_1=5K,\; R_2=3.33K,\; R_3=3K,\; R_4=18K}$
\begin{figure}[H]
	\centering
	\includegraphics[scale=0.5]{image/C1_simulate.png}
\end{figure}
\textbf{Nhận xét:}\\
- Với $V_1=4V$, $V_2=3V$ ,$V_3=2V$. Theo đề bài, $v_o=5V$ và kết quả mô phỏng được tại $v_o=5.008V$ đúng với yêu cầu thiết kế.\\

b. Giả sử cả 2 OPAMP có $V_{io}=100\mu V, I_{io}=4nA, I_{ib}=18nA$.\\
Xét tầng 1, ta có:
\[
\left\{
\begin{aligned}
I_{os} &= \lvert I^+ - I^- \rvert\\
I_{ib} &= \dfrac{I^+ + I^-}{2}
\end{aligned}
\right.
\;\Rightarrow\;
\left\{
\begin{array}{l}
\left[
\begin{aligned}
I^+ &= 20nA \\
I^- &= 16nA
\end{aligned}
\right. \\
\\
\left[
\begin{aligned}
I^+ &= 16nA \\
I^- &= 20nA \qquad \text{\hfill (chọn)}
\end{aligned}
\right.
\end{array}
\right.
\]
\[
\begin{aligned}
\Delta V \text{ do } V_{os} \text{ gây ra:} 
&\quad \pm \left( 1 + R_{f1}\left( \dfrac{1}{R_1} + \dfrac{1}{R_2} \right) \right)V_{os} = \pm 6V_{os} \\[6pt]
\Delta V \text{ do } I_{ib} \text{ gây ra:} 
&\quad \pm (I^- \cdot R_{f1}) = \pm 10K \cdot I^-
\end{aligned}
\]
Vậy tại ngõ ra tầng 1, $\boxed{v_{o1} = -2V_1-3V_2}$, $\boxed{\Delta V_1 = \pm 6V_{os} \pm 10K \cdot I^-}$\\

Xét tầng 2, ta có:
\[
\left\{
\begin{aligned}
I_{os} &= \lvert I^+ - I^- \rvert\\
I_{ib} &= \dfrac{I^+ + I^-}{2}
\end{aligned}
\right.
\;\Rightarrow\;
\left\{
\begin{array}{l}
\left[
\begin{aligned}
I^+ &= 20nA \\
I^- &= 16nA
\end{aligned}
\right. \\
\\
\left[
\begin{aligned}
I^+ &= 16nA \\
I^- &= 20nA \qquad \text{\hfill (chọn)}
\end{aligned}
\right.
\end{array}
\right.
\]
\[
\begin{aligned}
&\Delta V \text{ do } V_{os} \text{ gây ra:} 
\quad \pm \left( 1 + R_{f2}\left( \dfrac{1}{R_3} + \dfrac{1}{R_4} \right) \right)V_{os} = \pm 8V_{os} \\[6pt]
&\Delta V \text{ do } I_{ib} \text{ gây ra:} 
\quad \pm (I^- \cdot R_{f2}) = \pm 18K \cdot I^-\\
&\Delta V \text{ sai số } \Delta V_1 \text{ gây ra:}
\quad \pm 6V_{os} \pm 10K \cdot I^- \text{ (do độ lợi của $v_{o1}$ tại tầng 2 là -1 nên sai số không đổi) }
\end{aligned}
\]
\[
\rightarrow \Delta V_2 = \pm 8V_{os} \; \pm 18K \cdot I^- \; \pm 6V_{os} \; \pm 10K \cdot I^-
					   = \pm 1.96mV
\]

Vậy tại ngõ ra tầng 2, $\boxed{v_{o2} = 2V_1 + 3V_2 -6V_3 + \Delta V_2}$
