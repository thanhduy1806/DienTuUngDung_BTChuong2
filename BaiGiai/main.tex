\documentclass[11pt, a4paper, fleqn]{article}
\usepackage[utf8]{vietnam}
\usepackage{amsmath,amsfonts,amssymb}   %% AMS mathematics macros
\usepackage{graphics}
\usepackage{graphicx}
\usepackage{float}
\usepackage[top=2.5cm, bottom=2.5cm, left=2.5cm, right=2cm]{geometry}
\usepackage{setspace}
\usepackage{array}
\usepackage{booktabs}

\setlength{\parindent}{0pt} % bỏ thụt đầu dòng
\setlength{\mathindent}{0pt} % căn lề trái cho math
\makeatletter
\g@addto@macro\normalsize{
  \setlength{\abovedisplayskip}{1.5pt}
  \setlength{\belowdisplayskip}{1.5pt}
  \setlength{\abovedisplayshortskip}{1.5pt}
  \setlength{\belowdisplayshortskip}{1.5pt}
  \setlength{\jot}{1.5pt}
}
\makeatother
\setstretch{1.55}


%%%%% The Document

\begin{document}

\large
\begin{titlepage}
\begin{center}
\textbf{\large ĐẠI HỌC QUỐC GIA THÀNH PHỐ HỒ CHÍ MINH} \\
\textbf{\large TRƯỜNG ĐẠI HỌC BÁCH KHOA} \\
\textbf{\large KHOA ĐIỆN - ĐIỆN TỬ} \\
\textbf{\large BỘ MÔN ĐIỆN TỬ} 
\end{center}
\begin{figure}[h!]
\begin{center}
\includegraphics[width=6cm]{image/hcmut.png}
\end{center}
\end{figure}

\vspace{0.5cm}
\begin{center}
\begin{tabular}{c}
{\textbf{{\Large ĐIỆN TỬ ỨNG DỤNG - EE3129}}}\\\\
\hline \\
\textbf{\large BÀI TẬP CHƯƠNG 2} \\
\\
\hline \\
\end{tabular}
\end{center}

\vspace{0.5cm}

\begin{table}[h]
\begin{tabular}{rrlr}
\hspace{4 cm} & \large GVHD: & \large NGUYỄN TRUNG HIẾU \vspace{0.5cm}\\

& \large SV thực hiện: &\large VE SAMY & 2212901\\
&&\large NGUYỄN XUÂN PHÁT & 2212529\\
&&\large NGUYỄN THANH DUY & 2210527\\

\end{tabular}
\end{table}
\vspace{3cm}
\begin{center}
{\bf Thành phố Hồ Chí Minh, Tháng 10/2025}
\end{center}
\end{titlepage}

\begin{center}
\begin{tabular}{|>{\centering\arraybackslash}p{5cm}|>{\centering\arraybackslash}p{9cm}|}
\hline
\textbf{Họ và tên} & \textbf{Phân công bài tập} \\ 
\hline
Nguyễn Xuân Phát & Câu 1, 2, 3 \\ 
\hline
Nguyễn Thanh Duy & Câu 4, 5, 6, 10 \\ 
\hline
Ve Samy & Câu 7, 8, 9, 10 \\ 
\hline
\end{tabular}
\end{center}

\section{Câu 1}

\section{Câu 2}
Tìm công thức liên hệ giữa $I_L$ và $V_i$ (hình a) và giữa $I_L$ và $V_1 - V_2$ (hình b).

a. Giả sử OPAMP đang dùng là lí tưởng.

b. Giả sử cả 2 OPAMP có $V_{io}=100\,\mathrm{\mu V}$, $I_{io}=4\,\mathrm{nA}$, $I_{ib}=18\,\mathrm{nA}$.

Tìm ảnh hưởng của điều kiện không lý tưởng của OPAMP lên ngõ ra $I_L$.

\begin{figure}[H]
    \centering
    \includegraphics[scale=0.6]{image/De_C2.png} 
\end{figure}

\begin{center}
\textbf{Bài giải}
\end{center}


Đối với hình a:\\

\begin{figure}[H]
    \centering
    \includegraphics[scale=0.2]{image/C2_a1.png} 
\end{figure}

Xét nút $V^{-}$:
\begin{flalign*}
& \frac{V_i - V^-}{R_A} = -\frac{V_{o1} - V^-}{R_1} \quad(\text{theo chiều dòng}) && \\
& \Rightarrow V^- = \frac{V_i + V_{o1}}{2} \quad(\text{nếu }R_A=R_1) && 
\end{flalign*}

Tại $V^{+}$ (phân chia điện áp bởi hai trở bằng nhau):
\[
V^{+}=V_{o2}\cdot\frac{R_2}{R_2+R_2}=\frac{V_{o2}}{2}.
\]

Vì op-amp ở vùng phản hồi âm nên $V^{+}=V^{-}$. Do đó:
\[
\frac{V_{o2}}{2}=\frac{V_i+V_{o1}}{2}\quad\Rightarrow\quad V_i=V_{o2}-V_{o1}.
\]

Dòng tải theo $R_L$ (ghi theo ghi chú):
\[
I_L=\frac{V_i}{R_L}=\frac{V_{o1}}{R_L}=\frac{V_{o2}-V_i}{R_L}.
\]
\fbox{$\Rightarrow I_L=\frac{V_{o2}-V_i}{R_L}$}
\\
Đối với hình b:\\
\begin{figure}[H]
    \centering
    \includegraphics[scale=0.2]{image/C2_a2.jpg} 
\end{figure}

Xét OPAMP thứ nhất:
\begin{align*}
    \frac{V_1 - V_i}{R_1} &= \frac{V_{o1} - V_1}{R_2} \\
    V_1 \left(\frac{1}{R_1} + \frac{1}{R_2}\right) &= \frac{V_i}{R_1} + \frac{V_{o1}}{R_2}
\end{align*}

Xét OPAMP thứ hai:
\begin{align*}
    \frac{V_2 - V_i}{R_1} &= \frac{V_{o2} - V_2}{R_2} \\
    V_2 \left(\frac{1}{R_1} + \frac{1}{R_2}\right) &= \frac{V_i}{R_1} + \frac{V_{o2}}{R_2}
\end{align*}

Do đó:
\begin{align*}
    \frac{V_1}{R_1} + \frac{V_{o1}}{R_2} &= \frac{V_2}{R_1} + \frac{V_{o2}}{R_2} \\
    \frac{V_1 - V_2}{R_1} &= \frac{V_{o2} - V_{o1}}{R_2}
\end{align*}

Dòng điện qua tải:
\fbox{$\displaystyle
I_L = \frac{V_{o1} - V_{o2}}{R_x} = -\frac{(V_1 - V_2)R_2}{R_x \cdot R_1}
$}
\\[1em]
Câu b)\\
Đối với hình a:\\
\begin{figure}[H]
    \centering
    \includegraphics[scale=0.3]{image/C2_b1.png} 
\end{figure}

\begin{flalign*}
& V^- = V_{o1}\cdot\frac{R_1}{R_A + R_1} = \frac{V_{o1}}{2} && \\[1ex]
& V^+ = V_{o2} && \\[1ex]
& V^- = V^+ \Rightarrow \frac{V_{o1}}{2} = V_{o2} \Rightarrow V_{o1} = 2V_{o2} &&
\end{flalign*}

Xét tầng 2:\\
\begin{flalign*}
& V^+ = V_{o1} + V_{o2} = 3V_{o2} && \\[1ex]
& I_L = \frac{V^-}{R_L} = \frac{V^+}{R_L} = \frac{3V_{o2}}{R_L} &&
\end{flalign*}

\begin{flalign*}
\fbox{$\displaystyle I_L = \frac{3V_{o2}}{R_L}$}
\end{flalign*}
\\[1em]
Xét ảnh hưởng của dòng:\\
\begin{figure}[H]
    \centering
    \includegraphics[scale=0.4]{image/C2_b3.png} 
\end{figure}

Tại dòng của $I_b$ và $I_{os}$:
\begin{flalign*}
& \begin{cases}
I_b = \frac{I_1 + I_2}{2} \\[1ex]
I_{os} = |I_1 - I_2|
\end{cases} \Rightarrow
\begin{cases}
I^+ = 20\ \text{nA} \\
I^- = 16\ \text{nA} \\
I^+ = 16\ \text{nA} \\
I^- = 20\ \text{nA}
\end{cases} &&
\end{flalign*}

Ta có:
\begin{flalign*}
& V_A = -I^+ \cdot R_2//R_2 = -I^+ \cdot \frac{R_2}{2} && \\[2ex]
& I_1 = I^- + I_2 && \\[1ex]
& \Rightarrow I_1 = I^- + \frac{V_A}{R_1} && \\[1ex]
& = I^- + \frac{-I^+ \cdot R_2}{2R_1} && \\[2ex]
& \Delta V_1 - V_A = I_1 \cdot R_2 && \\[1ex]
& \Rightarrow \Delta V_1 = \left(I^- - \frac{I^+ \cdot R_2}{2R_1}\right)\cdot R_2 + V_A && \\[2ex]
& = \left(I^- - \frac{I^+ \cdot R_2}{2R_1}\right)\cdot R_2 - I^+ \cdot \frac{R_2}{2} && \\[2ex]
& \Rightarrow \Delta I_{L1} = \frac{\Delta V_1}{R_L} = \frac{\left(I^- - \frac{I^+\cdot R_2}{2R_1}\right)\cdot R_2 - I^+\cdot\frac{R_2}{2}}{R_L} &&
\end{flalign*}

\textbf{Xét tầng 2:}
\begin{flalign*}
& \Delta V_2 = (R_L\parallel R_3)\cdot (-I^-) && \\[2ex]
& \Delta I_{L2} = \frac{R_L\cdot R_3}{(R_L + R_3)\cdot R_L}\cdot (-I^-) && \\[2ex]
& \Delta I_{L2} = \frac{R_3}{R_L + R_3}\cdot (-I^-) && \\[3ex]
& \Rightarrow \Delta I = \Delta I_{L1} + \Delta I_{L2} = \frac{\left(I^- - \frac{I^+\cdot R_2}{2R_1}\right)\cdot R_2 - I^+\cdot\frac{R_2}{2}}{R_L} - \frac{R_3}{R_L + R_3}\cdot I^- && \\[2ex]
& \fbox{$\displaystyle \Delta I = \frac{\left(I^- - \frac{I^+\cdot R_2}{2R_1}\right)\cdot R_2 - I^+\cdot\frac{R_2}{2}}{R_L} - \frac{R_3}{R_L + R_3}\cdot I^-$}
\end{flalign*}
\\[1em]

Đối với hình b:\\
Xét ảnh hưởng do Vos:\\
\begin{figure}[H]
    \centering
    \includegraphics[scale=0.4]{image/C2_b4.png} 
\end{figure}

Xét $0 = V_{os} - \frac{V_{o1} - V_{o2}}{R_2}$:
\begin{flalign*}
& -V_{os}\cdot R_2 = R_1V_{os} - R_1V_{o1} && \\[1ex]
& \Rightarrow V_{o1} = V_{os}\cdot(R_1 + R_2) && \\[2ex]
& \Rightarrow \text{Ảnh hưởng của }V_{os}\text{ gây ra lỗi }\Delta V_{o1} = V_{os}(R_1 + R_2) && \\[2ex]
& \boxed{\Rightarrow \Delta I_L = \frac{\Delta V_{o1}}{R_x + Z_L} = \frac{V_{os}(R_1 + R_2)}{R_x + Z_L}} &&
\end{flalign*}
\\[1em]
\textbf{Khi ảnh hưởng của $I_b$ và $I_{os}$}\\
\textbf{Xét tầng 1:}
\begin{flalign*}
& V^+ = -I^+\cdot R_1\parallel R_2 && \\[2ex]
& I_1 = \frac{V^-}{R_1} = \frac{V^+}{R_1} = -\frac{I^+\cdot R_1\parallel R_2}{R_1} && \\[2ex]
& I_2 = I_1 + I^- && \\[2ex]
& \Rightarrow I_2 = -\frac{I^+\cdot R_1\parallel R_2}{R_1} + I^- && \\[2ex]
& \Delta V_{o1} - V^- = I_2\cdot R_2 && \\[2ex]
& \Rightarrow \Delta V_{o1} = \left(-\frac{I^+\cdot R_1\parallel R_2}{R_1} + I^-\right)\cdot R_2 + (-I^+\cdot R_1\parallel R_2) && \\[2ex]
& I_L\text{ do }\Delta V_{o1} = \frac{|\Delta V_{o1}|}{Z_L + R_x} && \\[2ex]
& \Rightarrow \Delta I_{L1} = \frac{\Delta V_{o1}}{Z_L + R_x} = \frac{\left(-\frac{I^+\cdot R_1\parallel R_2}{R_1} + I^-\right)\cdot R_2 + (-I^+\cdot R_1\parallel R_2)}{Z_L + R_x} &&
\end{flalign*}

\textbf{Xét tầng 2:}
\begin{flalign*}
& \Delta I_{L2} = \frac{R_L}{Z_L + R_L}\cdot (-I^-) && \\[3ex]
& \Rightarrow \Delta I_L = \Delta I_{L1} + \Delta I_{L2} = \frac{\left(-\frac{I^+\cdot R_1\parallel R_2}{R_1} + I^-\right)\cdot R_2 + (-I^+\cdot R_1\parallel R_2)}{Z_L + R_x} + \frac{R_L}{Z_L + R_L}\cdot (-I^-) &&
\end{flalign*}
\begin{center}
\fbox{$\displaystyle \Delta I_L = \frac{\left(-\frac{I^+\cdot R_1\parallel R_2}{R_1} + I^-\right)\cdot R_2 + (-I^+\cdot R_1\parallel R_2)}{Z_L + R_x} + \frac{R_L}{Z_L + R_L}\cdot (-I^-)$}
\end{center}

\section{Câu 3}

\section{Câu 4}
\begin{figure}[H]
      \centering
      \includegraphics[scale=0.5]{image/De_C4.png}
\end{figure}
\begin{flushleft}
\textbf{Cho mạch điện như hình trên} \\[4pt]
\textbf{a/ Giả sử OPAMP lý tưởng. Tìm hệ số $\dfrac{V_{O2}}{V_{in}}, \dfrac{V_{O1}}{V_{in}}, \dfrac{V_{O2}}{V_{O1}}$} \\[6pt]
\text{Xét Tầng 1, ta có:} \\[4pt]
Theo OPAMP lý tưởng: $V^+ = V^-$ và $i^+ = i^- = 0$ \\[4pt]
Mà $V^+ = 0(V) \Rightarrow V^- = 0(V) \Rightarrow V_A = 0(V)$ \\[4pt]
Có: $I_{in} = I_1 + I_2$ \\[8pt]

$\Rightarrow -\dfrac{V_{in}}{R} = \dfrac{V_{O1} - V_A}{10R} + \dfrac{V_{O2} - V_A}{10R}$ \\[6pt]
Vì $V_A = 0$, nên: \\[4pt]
$\Rightarrow -\dfrac{V_{in}}{R} = \dfrac{V_{O1}}{10R} + \dfrac{V_{O2}}{10R}$ \\[6pt]
$\Rightarrow -10V_{in} = V_{O1} + V_{O2}$ \quad (1) \\[10pt]

\text{Xét Tầng 2, ta có:} \\[4pt]
$V_{O2} = \left(1 + \dfrac{R}{R}\right) V_{O1} = 2V_{O1}$ \quad (2) \\[10pt]

\text{Từ (2) và (1), ta được:} \\[4pt]
$V_{O1} + 2V_{O1} = -10V_{in}$ \\[4pt]
$\Rightarrow 3V_{O1} = -10V_{in}$ \\[4pt]
$\Rightarrow \dfrac{V_{O1}}{V_{in}} = -\dfrac{10}{3}$ \\[10pt]


$\Rightarrow \dfrac{V_{O2}}{V_{in}} = 2 \cdot \dfrac{V_{O1}}{V_{in}} = -\dfrac{20}{3}$ \\[10pt]

\fbox{%
$\begin{aligned}
\dfrac{V_{O1}}{V_{in}} &= -\dfrac{10}{3} \\[6pt]
\dfrac{V_{O2}}{V_{in}} &= -\dfrac{20}{3}
\end{aligned}$} \\[10pt]
$\Rightarrow \fbox{$\dfrac{V_{O2}}{V_{O1}} = 2$}$
\end{flushleft}

\begin{figure}[H]
      \centering
      \includegraphics[scale=0.6]{image/C4a_vo1_vin.png}
      \includegraphics[scale=0.6]{image/C4a_vo2_vin.png}
      \includegraphics[scale=0.6]{image/C4a_vo2_vo1.png}
\end{figure}
\textbf{Nhận xét:} \text{giá trị ngõ ra khi mô phỏng có sự chênh lệch nhẹ nhưng không đáng kể với giá trị tính}
\text{được.}
\begin{flushleft}
\textbf{b/ Giả sử cả hai OPAMP có $V_{io}=12uV$, $I_{io}=12nA$, $I_{ib}=20nA$.} \\[8pt]
\textbf{Tìm ảnh hưởng của $V_{io}$ lên ngõ ra $V_o$.} \\[8pt]
\textbf{Tìm giá trị của R để có thể bỏ qua ảnh hưởng của dòng $I_{ib}$.} \\[8pt]     
\textbf{Xét ảnh hưởng $V_{OS}$ lên ngõ ra $V_o$}\\
\begin{figure}[H]
      \centering
      \includegraphics[scale=0.2]{image/C4b_vos_Iib.png}
\end{figure}
Xét tầng 1, ta có:\\[6pt]
Cho $V_{O2} = V_{in} = 0$\\[4pt]
$\Rightarrow V_{O1} = \left( 1 + \dfrac{10R}{10R \parallel R} \right) V_{OS}$\\[4pt]
$\Rightarrow V_{O1} = 12V_{OS} \quad (1)$\\[10pt]

Cho $V_{OS} = 0$\\[4pt]
$\Rightarrow V_{O1} = -\dfrac{10R}{10R}V_{O2} - \dfrac{10R}{R}V_{in}$\\[4pt]
$\Rightarrow V_{O1} = -(10V_{in} + V_{O2}) \quad (2)$\\[10pt]

Từ (1) và (2) $\Rightarrow V_{O1} = -(10V_{in} + V_{O2}) \pm 12V_{OS} \quad (3)$
\end{flushleft}

\begin{flushleft}
Xét tầng 2, ta có:\\[6pt]

Cho $V_{O1} = 0$\\[4pt]
$\Rightarrow V^+ = V^- = 0 \Rightarrow V_A = -V_{OS}$\\[4pt]
$\Rightarrow I^+ = I^- = I_{O1} = 0$\\[4pt]
$\Rightarrow I_1 = I_2$\\[4pt]
$\Rightarrow \dfrac{V_{O2} - V_A}{R} = \dfrac{V_A - 0}{R}$\\[4pt]
$\Rightarrow V_{O2} + V_{OS} = -V_{OS} \Rightarrow V_{O2} = -2V_{OS} \quad (1)$\\[10pt]

Cho $V_{OS} = 0$\\[4pt]
$\Rightarrow V_{O2} = \left( 1 + \dfrac{R}{R} \right) V_{O1}$\\[4pt]
$\Rightarrow V_{O2} = 2V_{O1} \quad (2)$\\[10pt]

Từ (1) và (2) $\Rightarrow V_{O2} = 2V_{O1} \pm 2V_{OS} \quad (4)$\\[4pt]
Từ (3) và (4): \\[4pt]
$\Rightarrow V_{O2} = -2\big[(10V_{in} + V_{O2}) \pm 12V_{OS}\big] \pm 2V_{OS}$\\[4pt]
$\Rightarrow V_{O2} = \dfrac{-20}{3}V_{in} \pm 8V_{OS}$\\[8pt]

$\boxed{V_{O2} = -\dfrac{20}{3}V_{in} \pm 8V_{OS}}$
\end{flushleft}

\begin{figure}[H]
      \centering
      \includegraphics[scale=0.6]{image/C4b_vo2_vin_vos.png}
\end{figure}
\textbf{Nhận xét:} \text{giá trị $v_o$ đo được bằng phần mềm mô phỏng gần đúng với giá trị tính toán.}


\begin{flushleft}
\textbf{Xét ảnh hưởng $I_{ib}$ và $I_{os}$ lên ngõ ra $V_o$}\\
\textbf{Xét tầng 1, ta có:}\\[4pt]
Cho $V_{O2} = V_{in} = 0$\\[4pt]
$\Rightarrow V^+ = V^- = 0$ (V)\\[4pt]
$\Rightarrow I_2 = \dfrac{V - 0}{10R} = 0$ (A)\\[4pt]
$\Rightarrow I_1 = I = \dfrac{V_{O1} - V}{10R}$\\[4pt]
$\Rightarrow V_{O2} = I \cdot 10R$ (A)\\[4pt]
Cho $I^+ = I^- = 0$ (A)\\[4pt]
$\Rightarrow V_{O1} = -(10V_{in} + V_{O2}) \quad (2)$\\[4pt]
Từ (1) và (2):\\[4pt]
$\Rightarrow V_{O1} = -(10V_{in} + V_{O2}) + I \cdot 10R \quad (3)$\\[10pt]

\textbf{Xét tầng 2, ta có:}\\[4pt]
Cho $V_{O1} = 0$\\[4pt]
$\Rightarrow V_{O2} = I \cdot R$\\[4pt]
Cho $I^+ = I^- = 0$\\[4pt]
$\Rightarrow V_{O2} = \left( 1 + \dfrac{R}{R} \right) V_{O1}$\\[4pt]
$\Rightarrow V_{O2} = 2V_{O1} \quad (4)$\\[4pt]
Từ (3) và (4):\\[4pt]
$\Rightarrow V_{O2} = 20V_{in} - 2V_{O2} \pm I \cdot 20R$\\[4pt]
$\Rightarrow V_{O2} = -\dfrac{20}{3}V_{in} \pm \dfrac{I \cdot 20R}{3}$\\[8pt]
Với $I = 26\,\text{nA}$ $\Rightarrow$ 
$\boxed{V_{O2} = -\dfrac{20}{3}V_{in} \pm 1{,}733 \times 10^{-7} R}$\\[10pt]

Để giảm tối thiểu ảnh hưởng $I_{os}$:\\[4pt]
$1{,}733 \times 10^{-7} R < 8V_{OS}$\\[4pt]
$\Rightarrow R < 553{,}85\,\Omega$
\end{flushleft}
\begin{figure}[H]
      \centering
      \includegraphics[scale=0.5]{image/C4b_Iib.png}
\end{figure}
\textbf{Nhận xét:} \text{chọn giá trị R=100 $\Omega$ cho thấy giá trị $V_o$ gần với tính toán hơn so với các giá trị }
\text{gần 500 $\Omega$}



\section{Câu 5}

\section{Câu 6}
\begin{flushleft}
Biên áp thay đổi: $10\,\text{mV} \rightarrow 30\,\text{mV}$\\[6pt]

\textbf{a) Thiết kế khuếch đại tín hiệu lên 100 lần}\\
Sử dụng khuếch đại không đảo:\\[6pt]
$V_o = \left( 1 + \dfrac{R_2}{R_1} \right) V_{in}$\\[4pt]
$\Rightarrow V_{out} = 100V_{in}$\\[4pt]
Chọn $R_2 = 99R_1$\\[10pt]

\textbf{b) Chọn linh kiện sao cho mạch hoạt động:}\\[4pt]
$V_{in}: \quad 10\,\text{mV} \le V_{in} \le 30\,\text{mV}$\\[4pt]
$V_{out}: \quad 1\,\text{V} \le V_{out} \le 3\,\text{V}$\\[6pt]

Sử dụng OPAMP OPA07CD có các giá trị hoạt động trong khoảng nhiệt độ từ $0^oC$ đến $70^oC$, tham khảo datasheet:\\[4pt]
$\begin{array}{l}
V_{OS} = 85\,\mu V\\
I_{ib} = 2{,}2\,\text{nA}\\
I_{OS} = 1{,}6\,\text{nA}\\
V_{OL} = -13\,\text{V}, \quad V_{OH} = +13\,\text{V} \quad (\text{với } R_L = 10\,k\Omega)
\end{array}$\\[10pt]

$\Rightarrow$ Có $V_{OS}$ rất nhỏ, không ảnh hưởng nhiều đến ngõ ra.\\[4pt]

\end{flushleft}

\begin{figure}[H]
      \centering
      \includegraphics[scale=0.15]{image/C6b.png}
\end{figure}

\begin{flushleft}
\[
\begin{cases}
|I^+ - I^-| = 1{,}6\,\text{nA}\\
I^+ + I^- = 4{,}4\,\text{nA}
\end{cases}
\Rightarrow
\begin{cases}
I^+ = 1{,}4\,\text{nA}\\
I^- = 3\,\text{nA}
\end{cases}
\Rightarrow I_b = 3\,\text{nA}
\]\\[8pt]

\textbf{Xét ảnh hưởng do $V_{OS}$ lên ngõ ra}\\
Dùng định lý xếp chồng:\\[4pt]
Cho $V_{OS} = 0$\\[4pt]
$\Rightarrow V_{out} = 100V_{in}$ \hfill (1)\\[6pt]
Cho $V_{in} = 0$\\[4pt]
$I_R = I_O \quad (\text{vì } I^+ + I^- = 0)$\\[4pt]
$V_O = -100V_{OS}$ \hfill (2)\\[6pt]
Từ (1) và (2): $V_{out} = 100V_{in} \pm 100V_{OS}$\\[10pt]

\textbf{Xét ảnh hưởng $I_b$ và $I_{OS}$ lên ngõ ra}\\
Dùng định lý xếp chồng:\\[4pt]
Cho $I^+ + I^- = 0$\\[4pt]
$\Rightarrow V_{out} = 100V_{in}$ \hfill (3)\\[6pt]
Cho $V_{in} = 0$\\[4pt]
$\Rightarrow V_{out} = I_b \cdot R = 3\,\text{nA} \cdot R = 3nR$ \hfill (4)\\[6pt]
Từ (3) và (4): $V_{out} = 100V_{in} \pm 3nR$\\[8pt]

Để giảm ảnh hưởng $I_b$ và $I_{OS}$ lên ngõ ra:\\[4pt]
$3nR < 100V_{OS}$\\[4pt]
$\Rightarrow R < 2{,}8\,\text{M}\Omega$ \hfill (5)\\[4pt]
Chọn $R = 10\,k\Omega$\\[10pt]

\begin{figure}[H]
      \centering
      \includegraphics[scale=0.6]{image/C6.png}
\end{figure}

\textbf{Nhận xét:}\\[4pt]
Ảnh hưởng của $V_{OS}$ và $I_b, I_{OS}$ đến ngõ ra là nhỏ.\\
Giá trị $R = 10\,k\Omega$ đảm bảo mạch hoạt động ổn định so với các điện trở có giá trị cao hơn.
Khi gắn tải $R_L$, thì hệ số $A_v$ vẫn đạt gần như ổn định, hầu như không có sai số đáng kể.\\[4pt]
Không cần sử dụng 2 chân điều chỉnh \textit{offset} của OPAMP với mạch này.
Độ lợi tại ngõ ra khi mô phỏng xấp xỉ 100 gần bằng với yêu cầu của đề bài đưa ra
\end{flushleft}


\section{Câu 7}

\section{Câu 8}

\section{Câu 9}
Cho một tín hiệu cần xử lý ở dạng điện áp, có tầm thay đổi từ 40mV – 200mV.\\
a. Thiết kế mạch cho ngõ ra tầm 4mA – 20mA.\\
b. Lựa chọn OPAMP và các linh kiện cần thiết để mạch hoạt động. Lưu ý: xử lý các
thông số không lý tưởng của OPAMP. (Tham khảo datasheet)\\

\begin{center}
\textbf{Bài giải}
\end{center}

a. Để thiết kế mạch chuyển đổi tín hiệu điện áp sang dòng điện ta biểu diễn mối quan hệ giữa dòng điện và điện áp như hình bên dưới
\begin{figure}[H]
    \centering
    \includegraphics[scale=0.25]{image/C9_chart.png} 
\end{figure}
Từ hình trên ta có hệ phương trình:
\begin{equation*}
    \begin{cases}
        4 = 40\cdot a + b  \\
        20 = 200\cdot a + b\\
    \end{cases}
\end{equation*}
Giải hệ phương trình ta được:  
\begin{equation*}
    \begin{cases}
        a = 0.1 \\
        b = 0 \\
    \end{cases}
\end{equation*}
Vậy ta có phương trình mạch chuyển đổi điện áp sang dòng điện như hình bên dưới:\\
\begin{equation*}
    \boxed{I = 0.1V = \dfrac{V}{10\Omega}}
    \rightarrow \text{ Chọn } R_3 = 10\Omega
\end{equation*}
\\
b/ Ta phải chọn tỉ lệ
\begin{equation*}
    \dfrac{R_1}{R_5} = \dfrac{R_4}{R_3}
\end{equation*}
Chọn các điện trở khác đều là $10k\Omega$ để phù hợp với tỉ lệ và tận dụng thông số trở.\\
Chọn OPAMP LM358P vì các thông số tham khảo được: \\
- Băng thông: 1MHz, phù hợp cho mạch chuyển đổi tín hiệu chậm.\\
- Slew Rate: 0.3V/$\mu s$, phù hợp cho mạch so sánh và chuyển mạch chậm.\\
- Độ trễ ngõ ra so với ngõ vào thấp, phù hợp cho mạch tạo xung ổn định.\\
- Vos (độ dịch chuyển ngõ vào): 2mV.\\
- Ib (dòng dịch chuyển ngõ vào): 2nA.\\
- Ios (dòng dịch chuyển bù ngõ vào): 20nA (max).\\
Mạch hoàn chỉnh như hình bên dưới:
\begin{figure}[H]
    \centering
    \includegraphics[scale=0.4]{image/C9.png}
\end{figure}


\section{Câu 10}

\section{Câu 11}

\section{Câu 12}

\section{Câu 13}
Cho mạch điện như ở Hình 8a. OPAMP được cấp nguồn có $V_{OH}=3.3V$, $V_{OL}=-3V$.
Các điện trở được chọn $R_1=2K$, $R_2=3K$, $R_3=4K$, $R_4=6K$.\\
Cho dạng sóng Vi(t) như ở Hình 8b. Vẽ dạng sóng ngõ ra Vo trong các trường hợp sau:\\
a. $V_1=2V$, $V_{m1}=2V$, $V_{m2}=-3V$\\
b. $V_1=4V$, $V_{m1}=1V$, $V_{m2}=-3V$
\begin{figure}[H]
	\centering
	\includegraphics[scale=0.9]{image/C13_De.png}
\end{figure}
\begin{center}
\textbf{Bài giải}
\end{center}
a. Giả sử OPAMP lý tưởng:\\
Ta có:
\[
\left\{
\begin{aligned}
V^- &= \dfrac{R_2}{R_1+R_2}V_i \\
V^+ &= \dfrac{\dfrac{V_1}{R_3} + \dfrac{V_o}{R_2}}{\dfrac{1}{R_3} + \dfrac{1}{R_4}} = \dfrac{3}{5}V_1 +\dfrac{2}{5}V_o
\end{aligned}
\right.
\]

\[
\begin{aligned}
&\text{Giả sử }& V_o = V_{OH} &\qquad\rightarrow\qquad V^+ = \dfrac{3}{5}\cdot2 + \dfrac{2}{5}\cdot3.3 \quad;\quad V^- = 0.6V_i \\
&\text{Để }& V_o = V_{OL} &\qquad\rightarrow\qquad V^+ < V^- \quad\rightarrow\quad \boxed{V_i>4.2V}\\
&\rightarrow& V_o = V_{OL} &\qquad\rightarrow\qquad V^+ = \dfrac{3}{5}\cdot2 + \dfrac{2}{5}\cdot-3 \quad;\quad V^- = 0.6V_i \\
&\text{Để }& V_o = V_{OH} &\qquad\rightarrow\qquad V^+ > V^- \quad\rightarrow\quad \boxed{V_i<0V}\\
\end{aligned}
\]\\
Tại t=0 đến khi $V_i(t)=0$ nằm trong khoảng Deadband của OPAMP. Vì thế tùy vào trạng thái ban đầu sẽ cho ra ngõ ra khác nhau.
Sau khi $V_i(t)<0$ thì $V_o = V_{OH}$ và $V_i$ luôn bé hơn 4.2V nên không thể cho $V_o=V_{OL}$. Vì thế nên $V_o$ từ đó luôn bằng $V_{OH}$
\begin{figure}[H]
	\centering
	\includegraphics[scale=1]{image/C13_a_BT.png}
\end{figure}

\begin{figure}[H]
    \centering
    \begin{subfigure}[b]{0.48\textwidth}
        \centering
        \includegraphics[width=\textwidth]{image/C13_a.png}
        \caption*{Hình a1}
    \end{subfigure}
    \hfill
    \begin{subfigure}[b]{0.48\textwidth}
        \centering
        \includegraphics[width=\textwidth]{image/C13_a_1.png}
        \caption*{Hình a2}
    \end{subfigure}
\end{figure}

\textbf{Nhận xét:}\\
- Hình a1 mô phỏng $V_i$ theo đề bài với màu xanh, $V_o=-3.380V$ màu đỏ xấp xỉ 3.3V đúng với tính toán.\\
- Hình a2 chọn $V_i(t)$ thỏa $V_i>V_{UT}$ và $V_i<V_{LT}$ để kiểm tra lại $V_{UT}=4.288V$ và $V_{LT}=-2.970V$. 
Theo kết quả mô phỏng được có thể thấy $V_{UT}$ xấp xỉ 4.2V, còn $V_{LT}$ do độ dốc lớn nên không thể đo chính xác được.\\

b. Giả sử OPAMP lý tưởng:\\
Ta có:
\[
\left\{
\begin{aligned}
V^- &= \dfrac{R_2}{R_1+R_2}V_i \\
V^+ &= \dfrac{\dfrac{V_1}{R_3} + \dfrac{V_o}{R_2}}{\dfrac{1}{R_3} + \dfrac{1}{R_4}} = \dfrac{3}{5}V_1 +\dfrac{2}{5}V_o
\end{aligned}
\right.
\]

\[
\begin{aligned}
&\text{Giả sử }& V_o = V_{OH} &\qquad\rightarrow\qquad V^+ = \dfrac{3}{5}\cdot4 + \dfrac{2}{5}\cdot3.3 \quad;\quad V^- = 0.6V_i \\
&\text{Để }& V_o = V_{OL} &\qquad\rightarrow\qquad V^+ < V^- \quad\rightarrow\quad \boxed{V_i>6.2V}\\
&\rightarrow& V_o = V_{OL} &\qquad\rightarrow\qquad V^+ = \dfrac{3}{5}\cdot4 + \dfrac{2}{5}\cdot-3 \quad;\quad V^- = 0.6V_i \\
&\text{Để }& V_o = V_{OH} &\qquad\rightarrow\qquad V^+ > V^- \quad\rightarrow\quad \boxed{V_i<2V}\\
\end{aligned}
\]\\
Tại t=0 đến khi $V_i(t)=0$ nằm trong khoảng Deadband của OPAMP. Vì thế tùy vào trạng thái ban đầu sẽ cho ra ngõ ra khác nhau.
Sau khi $V_i(t)<0$ thì $V_o = V_{OH}$ và $V_i$ luôn bé hơn 6.2V nên không thể cho $V_o=V_{OL}$. Vì thế nên $V_o$ từ đó luôn bằng $V_{OH}$
\begin{figure}[H]
	\centering
	\includegraphics[scale=1]{image/C13_b_BT.png}
\end{figure}

\begin{figure}[H]
    \centering
    \begin{subfigure}[b]{0.48\textwidth}
        \centering
        \includegraphics[width=\textwidth]{image/C13_b.png}
        \caption*{Hình b1}
    \end{subfigure}
    \hfill
    \begin{subfigure}[b]{0.48\textwidth}
        \centering
        \includegraphics[width=\textwidth]{image/C13_b_1.png}
        \caption*{Hình b2}
    \end{subfigure}
\end{figure}

\textbf{Nhận xét:}\\
- Hình a1 mô phỏng $V_i$ theo đề bài với màu xanh, $V_o=-3.383V$ màu đỏ xấp xỉ 3.3V đúng với tính toán.\\
- Hình a2 chọn $V_i(t)$ thỏa $V_i>V_{UT}$ và $V_i<V_{LT}$ để kiểm tra lại $V_{UT}=6.294V$ và $V_{LT}=-2.944V$. 
Theo kết quả mô phỏng được có thể thấy $V_{UT}$ xấp xỉ 6.2V, còn $V_{LT}$ do độ dốc lớn nên không thể đo chính xác được.\\

\section{Câu 14}



\end{document}